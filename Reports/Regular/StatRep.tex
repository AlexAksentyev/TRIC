\documentclass{article}
\usepackage{../repsty}
\usepackage{threeparttable}
\usepackage{url}
\usepackage{hyperref}
\newcommand{\Dim}[1]{\mathrm{#1}}




\begin{document}

\title{Statistical modeling}
\maketitle

\section*{Introduction}
\dots
Since the estimate $\Ayy*$ is obtained as a difference statistic of the slopes in the up- and down-polarized trials, its variance $\SE{\Ayy*} = C\sqrt{2}\cdot\SE{\slp*}$, with the proportionality coefficient $C = (\nu\CS[0]\Thick P^t\DP)^{-1} \approx 1.3\cdot 10^{5}$. 

For the mean, 
\begin{equation}\label{eq:AvgAyySE}
	\SE{\avg{\Ayy*}} = \frac{\SE{\Ayy*}}{\sqrt N} = \sqrt{2}\sqrt{\frac hH}\cdot\SE{\Ayy*},
\end{equation}
where $H$ is the beam time, $h$ the cycle length, and so the number of estimate pairs $N = \sfrac{H}{2h}$.

\begin{table}
\centering
\caption{Parameter values (June 2016)\label{tbl:Param}}
\begin{threeparttable}[h]
\begin{tabular}{llr}
\hline\hline
Parameter					& Value 				& Dimension \\
\hline
$\nu$						& 0.79 					& MHz \\
$\Thick$					& $1.1\cdot 10^{14}$	& $\Dim{at\cdot cm^{-2}}$ \\
$P^t$						& 0.88					& -- \\
$\DP$						& 1.48					& -- \\
$\CS[0]$\tnote{a}			& 70					& mb\\
\hline
\end{tabular}
\begin{tablenotes}
	\item[a]{From Particle Data Grpoup \url{http://pdg.lbl.gov/2016/hadronic-xsections/rpp2014-pd_pn_plots.pdf}}
\end{tablenotes}
\end{threeparttable}
\end{table}



\section{Necessary beam time}
\newcommand{\Tint}{\Delta t}
\DeclareDocumentCommand{\v}{m}{\sigma^2\bkt*{#1}}

Under the Gauss-Markov conditions, the slope estimate's standard error is
\begin{equation}\label{eq:SlpVarGM}
	\SE{\slp*} = \frac{\SE{\err}}{\sqrt{\sum_k (t_k - \avg{t})^2}}.
\end{equation}

Since the measurements are taken uniformly in time with the step $\Tint$, rewriting eq.~\eqref{eq:SlpVarGM} in physical terms gets:
\begin{align*}
	\sum_{k=1}^K (t_k - \avg{t})^2 &= \sum_k (k\Tint - \frac{1}{K}\sum_{k=1}^K k\Tint)^2; \\
	\frac{1}{K}\sum_{k=1}^K k\Tint  &= \frac{\Tint}{2}(K+1)\equiv \Tint\mu, \\
	\sum_k (k\Tint - \mu\Tint)^2 	&= \Tint^2\sum_k\bkt{
											k^2 - 2k\mu + \mu^2
										} \\
									&= \Tint^2\bkt{\sum_k k^2 - 2\mu\sum_k k + \mu^2K} \\
									&= \Tint^2\bkt{
											\frac{2K+1}{3}K\mu - 2\mu^2K + \mu^2K
										} 
		 							 = \Tint^2\mu K\bkt{\frac{2K+1}{3} - \mu} \\
									&= \Tint^2\mu K \frac{K-1}{6} \\
									&= \frac{\Tint^2}{12}K(K^2-1),
\shortintertext{and hence}									
	\sqrt{\sum_{k=1}^K(t_k - \avg{t})^2} = \frac{\Tint}{2\sqrt{3}}\sqrt{K}\sqrt{K^2-1}.
\end{align*}
As the number of measurements grows, $K \gg 1$, $K^2-1\approx K^2$, and so
\[
	\sqrt{\sum_k (t_k - \avg{T})^2} \approx \frac{\Tint}{2\sqrt{3}}K\sqrt{K}.
\]

The number $K$ of measurements that go into evaluating a slope is related to the state time $h$ as $K\Tint = h$, hence
\[
	\sqrt{\sum_{k=1}^K (t_k - \avg{t})^2} \approx \frac{h\sqrt{h}}{2\sqrt{3}\sqrt{\Tint}}.
\]
Finally, for the slope variance,
\[
	\SE{\slp*} = 2\sqrt{3}\sqrt{\frac{\Delta t}{h}}\frac{\SE{\err}}{h}.
\]

By plugging it in eq.~\eqref{eq:AvgAyySE} we get
\begin{equation}\label{eq:AvgAyySEPhys}
\SE{\avg{\Ayy*}} = 4\sqrt{3}C\cdot \frac{\sqrt{\Tint}}{h\sqrt{H}}\cdot\SE{\err}.
\end{equation}

\begin{figure}[h]
\centering
\includegraphics[scale=1]{BeamTime_15minCycle}
\caption{Beam time (in full days) as a function of the standard error of the mean $\Ayy$ estimate, required in the case of 15-minute cycles.}
\end{figure}

\section{Necessary cycle length}



\end{document}