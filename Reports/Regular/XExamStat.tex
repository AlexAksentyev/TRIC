\documentclass{article}
\usepackage[left=1cm,top=1cm,right=1cm,bottom=1cm]{geometry}

\usepackage{mathtools}
\usepackage{amsmath}
\usepackage{subcaption}

\usepackage{xparse}

\usepackage[standard,amsmath,thmmarks,hyperref,thref]{ntheorem}	

\theoremseparator{.}
\theorembodyfont{\normalfont}
\theoremsymbol{\ensuremath{\qed}}
\renewtheorem{Definition}{Def.}

\newcommand{\err}{\epsilon}
\DeclareDocumentCommand{\CS}{sO{}}{\IfBooleanTF{#1}{\hat{\sigma}_{#2}}{\sigma_{#2}}}
\DeclareDocumentCommand{\slp}{sO{}}{\IfBooleanTF{#1}{\hat{\beta}_{#2}}{\beta_{#2}}}
\DeclareDocumentCommand{\Thick}{O{}}{\Theta_{#1}}

\DeclareDocumentCommand{\SE}{sm}{\IfBooleanTF{#1}{\hat{\sigma}\bkt*{#2}}{\sigma\bkt*{#2}}}

\DeclareDocumentCommand{\bkt}{sm}{\IfBooleanTF{#1}{\left[ #2 \right]}{\left(#2\right)}}
\newcommand{\td}{\mathrm{d}}

\DeclareDocumentCommand{\Ayy}{sO{}}{\IfBooleanTF{#1}{\hat{A}_{y,y}}{A_{y,y}}^{#2}}
\DeclareDocumentCommand{\D}{s}{\IfBooleanTF{#1}{\hat{D}}{D}}
\DeclareDocumentCommand{\R}{s}{\IfBooleanTF{#1}{\hat{R}}{R}}
\DeclareDocumentCommand{\DP}{s}{\IfBooleanTF{#1}{(P^+ - P^-)}{\Delta P}}
\DeclareDocumentCommand{\SP}{s}{\IfBooleanTF{#1}{(P^+ + P^-)}{\Sigma P}}

\newcommand{\congd}{\overset{d}{\cong}}

\begin{document}

\section{$\Ayy$ estimators}

Assuming the model for the slope of $\ln I_t = \ln I_0 + \slp t + \err_t$ is 
\[
\slp = -\nu\CS[X]\Thick[X] - \nu\CS[0]\Thick\bkt{1 + PP^t \Ayy},
\]
two ways to construct the estimator for $\Ayy$ are via the slope difference $\D = \slp^- - \slp^+ = \nu\CS[0]\Thick P^t \DP* \Ayy$, with the corresponding \emph{difference} estimator 
\[
	\Ayy*[D] = \bkt{\nu\CS[0]\Thick}^{-1}\cdot\frac{\D*}{P^t\DP},
\]
or the slope ratio $\delta = \slp^+/\slp^-$, and the R-statistic
\[
	\R = \frac{\delta - 1}{\delta + 1} = \frac{\DP* P^t\Ayy}{2\cdot\bkt{1 + x} + \SP* P^t\Ayy}, ~x = \frac{\CS[X]\Thick[X]}{\CS[0]\Thick},
\]
with
\[
	\Ayy*[R] = \frac{2\R*}{P^t\bkt{\DP -\R*\cdot\SP}}\cdot\bkt{1+x}.
\]

\section{Comparison of the estimators}
\begin{Definition}[Congruent in distribution]
\emph{Congruent in distribution} means 
\[
	X \congd Y \Leftrightarrow f(X|\theta) = f(g(Y)|\theta).
\]
The transformation $g$ is called the d-congruency transformation.
\end{Definition}
We can compare whether $\Ayy*[D]$ and $\Ayy*[R]$ are congruent in distribution by using the QQ-plot. Here it is:
\begin{figure}[h]
\centering
\begin{subfigure}{.5\textwidth}
\includegraphics[scale=.5]{R_D_QQPlot}
\caption{Red line $y = a + b\cdot x$.}
\end{subfigure}~
\begin{subfigure}{.5\textwidth}
\includegraphics[scale=.5]{R_gD_QQPlot}
\caption{Red line $y = x$.}
\end{subfigure}
\caption{QQ-Plot for the sample quantiles of the R vs D statistics.}
\end{figure}

If the distributions of the D- and R-stats are linearly related, the points of the QQ-plot will lie on a line. The transformation $g(\D*) = \alpha + \beta\D* = \frac{2\R*}{P^t\bkt{\DP - \R*\SP}}$.

This means that
\begin{equation*}
\begin{cases}
	\Ayy*[R] &= g(\D*)\cdot(1+x), \\
	\D* 	 &= \Ayy*[D]\cdot\bkt{P^t\DP\nu\CS[0]\Thick};
\end{cases}
\end{equation*}
hence
\[
	\Ayy*[R] = (1+x) \cdot \bkt*{\alpha + \beta\cdot\bkt{P^t\DP\nu\CS[0]\Thick}\cdot\Ayy*[D]}
\]

%\section{Estimation of $x$}
%
%Assume the following model:
%\begin{equation*}
%\begin{cases}
%	\slp[0] &= -\nu\CS[X]\Thick[X], \\
%	\slp[1] &= -\nu\CS[X]\Thick[X] - \nu\CS[0]\Thick\bkt{1 + PP^t\Ayy}.
%\end{cases}
%\end{equation*}
%
%Then, we can write the model $\slp[1] = \alpha + \beta\slp[0] + \err$, where $\beta$ should be 



\end{document}